\documentclass{article}

\usepackage{xltxtra} %% For XeTeX font commands
\usepackage[includeheadfoot, margin = 0.5in]{geometry} %% Margins
\usepackage[pdftitle = {POC County Report},
            pdfauthor = {Partners for Our Children},
            hidelinks,
            unicode = true]
{hyperref}
\usepackage{graphicx} 
\usepackage{sectsty} %% Change format (font) of section headers
\usepackage{tikz}    %% Graphics for banner
\usepackage{parskip} %% Lines between paragraphs, no indentation
\usepackage{booktabs} %% Pretty up the tables
\usepackage{xcolor}

%% Colors
\definecolor{pocDGreen}{HTML}{788172}
\definecolor{pocLGreen}{HTML}{A2B69A}
\definecolor{pocDBlue}{HTML}{3B6E8F}
\definecolor{pocLBlue}{HTML}{A3DCE6}
\definecolor{pocPurple}{HTML}{A784B4}

%% Fonts
\setmainfont{Frutiger LT Std 55 Roman}
\allsectionsfont{\fontspec{Archer}}

\usepackage{fancyhdr} %% Header and Footer formatting
\pagestyle{fancy}

%%% Header
\renewcommand{\sectionmark}[1]{\markboth{\MakeUppercase{#1}}{\MakeUppercase{#1}}}
\fancyhf{}
\renewcommand{\headrulewidth}{0.5pt}
\renewcommand{\headrule}{\hbox to\headwidth{%
\color{pocDGreen}\leaders\hrule height \headrulewidth\hfill}}
\rhead{\color{pocDGreen} \leftmark}

%%% Footer
\lfoot{\color{pocDGreen} \href{http://www.partnersforourchildren.org}{www.partnersforourchildren.org}}
\rfoot{\color{pocDGreen} \thepage}
\renewcommand{\footrulewidth}{0.5pt}
\renewcommand{\footrule}{\hbox to\headwidth{%
\color{pocDGreen}\leaders\hrule height \footrulewidth\hfill}}

\begin{document}






\lhead{\color{pocDGreen} Pierce County Report}
\thispagestyle{empty} %% No header/footer on first page

\begin{tikzpicture}[x=1in, y=1in]

    %%    Set up constants
    \def\banX{\textwidth}
    \def\banY{3.3in}
    \def\stripeHeight{0.5in}
    \def\stripeYpos{0.55in} %% From top
    \def\triX{0.25in}
    \def\triY{0.15in}
    \def\logoInsetX{0.7in}
    \def\logoInsetY{18pt}
    
    %% Draw Background Geomety
    \filldraw[pocLGreen] (0, 0) rectangle ++(\banX, \banY);
    \filldraw[pocDGreen] (0, \banY - \stripeYpos) rectangle ++(\banX + \triX, - \stripeHeight);
    \filldraw[fill=pocLGreen, draw=pocDGreen, join=bevel, thick]
        (\banX, \banY - \stripeYpos - \stripeHeight) -- ++(\triX, 0) -- ++(-\triX, -\triY) -- cycle;
    
    %% Above-stripe Text
    \node[pocDBlue, below left = 6pt, align = right] at (\banX, \banY)
        {\textbf{Automated County Report}\\\textbf{Generated \today}};
    \node at (\logoInsetX, \banY - \logoInsetY)
        {\includegraphics[height=0.3in]{pocLogoSmall}};
    
    %% Stripe Text
    \node[right = 6pt, white] at (0, \banY - \stripeYpos - \stripeHeight + 16pt)
        {\fontspec{Archer}\Huge{Focus on Pierce County}};
    
    %% Below Stripe Text
    \node[below right = 6pt, pocDBlue, align = left, text width = \textwidth - 12pt]
        at (0, \banY - \stripeYpos - \stripeHeight - 3pt) {
        This is a DRAFT VERSION for a Partners for Our Children automatically generated county report.
        The reports will be generated for any/every county; this sample report will focus on Pierce County.\\[6pt]
        
        Much like the Data Portal itself, the default report has three sections: Investigations \& Assessments,
        In-Home Services, and Out-of-Home Care. As a starting point, divide each of these into two parts,
        (1) \emph{County Focus}, a detailed Trends-style recent history specific to the focus county, and
        (2) \emph{Regional Context}, an attempt to put the county in context by comparing it to other counties in the same region.
        Of course, adding more is possible--safety measures and the other categories that come with Out-of-Home Care would be easy to tack on.
        You will notice that Investigations \& Assessments and In-Home Services use offices and office goups, while Out-of-Home Care uses counties. This mirrors the Data Portal and is necessitated by the level of detail of our data extracts from Children's Administration. We have a Technical Bulletin in progress that fully explains this.\\[6pt]
        
        \textbf{Note:} Please excuse any awkward spacing and pagebreaks. This report was created by a computer. 
    };
    
\end{tikzpicture}

\vspace{12pt}

\section*{Overview}

In 

{\ttfamily\noindent\bfseries\color{errorcolor}{\\Error in eval(expr, envir, enclos) : object 'context\_year' not found}}, Pierce County had {

{\ttfamily\noindent\bfseries\color{errorcolor}{\\Error in formatC(focus\_pop\_person, big.mark = ",", format = "d") : \\\ \ object 'focus\_pop\_person' not found}}} children under age 18 and {

{\ttfamily\noindent\bfseries\color{errorcolor}{\\Error in formatC(focus\_pop\_house, big.mark = ",", format = "d") : \\\ \ object 'focus\_pop\_house' not found}}} households with children. These are the statistics used in the denominators for the rates in the \emph{Regional Context} sections below. The Department of Social and Health Services divides the state into 3 regions. This report will put Pierce County data in context by also including state-wide data as well as data from the rest of of the region. The counties in Region 

{\ttfamily\noindent\bfseries\color{errorcolor}{\\Error in eval(expr, envir, enclos) : object 'region\_cd' not found}} are...\\

If you are viewing this document on an Internet-connected device, you can click on the titles to visit the corresponding sections of the Child Well-Being Data Portal. The graphs should be fairly intuitive. In the Focus Graphs that lead each section, an upper bar graph shows a point-in-time count corresponding to the section it's in. These point-in-time counts are taken on the first day of each month. Below each bar graph is an activity graph that shows the number of opened cases or children entering care as positive numbers, and the number of closed cases or children exiting care as negative numbers. Individual points are plotted indicating the activity measures for each month, and the general trend is captured in a \textsc{LOESS}-smoothed regression line.
\textsc{LOESS} is a data smoothing method where nearby points carry greater weight.
The net change for each month is plotted as a thin vertical red line.\\

The \emph{Regional Context} sections all feature a "dotplot" with comparisons to other counties/office groups in the same region. It is important to note that these use rates (per 1,000) rather than absolute counts. Using rates is the preferred way to compare communities of different sizes.

The table below gives an overview of more the more general county characteristics.




\section{\href{http://www.partnersforourchildren.org//child-well-being/visualizations/investigations-assessments/trends}
{Investigations \& Assessments}}
When professionals and community members report suspected instances of child abuse or neglect to the child welfare system, some of the reports are investigated, some are assessed only (e.g., Family Reconciliation Services), and some \emph{screened out} because the information reported (if true) does not meet the statutory definition of child abuse or neglect and there is no need for an assessment.

The measurements in this section provide an overview of the changes over time in the number of the Washington State households who have received investigations and/or assessments.

\subsection{\href{http://www.partnersforourchildren.org//child-well-being/visualizations/investigations-assessments/trends}
{Investigations \& Assessments:} Pierce County Focus}
These graphs show the recent trends in Investigations \& Assessments for the DCFS offices in
Pierce County.
















